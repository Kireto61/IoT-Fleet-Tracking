\begin{abstract}
    В настоящата курсова работа е разработена интелигентна система за управление на автопарк с IoT интеграция, базирана на нерелационна база данни MongoDB. Проектът демонстрира практически подход към решаването на съвременни предизвикателства в логистичната индустрия, свързани с обработката на големи обеми данни от IoT сензори в реално време.

    Системата включва три основни колекции: превозни средства, товарни пратки и телеметрични данни от сензори. Реализирани са всички задължителни CRUD операции с различни видове филтриране, агрегиращи заявки за анализ на данни и механизъм за управление на достъпа чрез роли (RBAC). Проектът съдържа над 45 реалистични записа и демонстрира ефективното използване на MongoDB за работа с IoT данни.

    Разработен е REST API сървър с Node.js и Express.js, както и интерактивен уеб интерфейс за визуализация на данните. Системата е контейнеризирана с Docker и включва автоматизирани тестове и CI/CD pipeline.

    Проектът доказва предимствата на нерелационните бази данни при работа с динамични IoT данни и предоставя солидна основа за бъдещо разширение на логистични платформи.
\end{abstract}