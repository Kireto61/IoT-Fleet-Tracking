% ============================================================================
% Глава 1: Въведение
% ============================================================================

\section{ВЪВЕДЕНИЕ}

В съвременната логистична индустрия управлението на флотилии от превозни средства става все по-сложно поради нарастващия обем на данни от IoT сензори и необходимостта от реално време мониторинг. Традиционните системи за управление на флоти се сблъскват с предизвикателства при обработката на големи количества динамични данни от сензори, GPS тракери и системи за диагностика.

Бизнес проблемът, който решава този проект, е ефективното съхранение и обработка на данни от IoT устройства в системата за управление на флотилии. Системата трябва да проследява състоянието на превозните средства, данните от сензори в реално време, графици за поддръжка и история на ремонта. Това изисква гъвкава архитектура, която може да се адаптира към различни видове сензори и променящи се изисквания без често преструктуриране.

Основните цели на проекта включват:
\begin{enumerate}
    \item Създаване на работеща база данни в MongoDB с правилно структурирани данни за IoT флот управление
    \item Реализация на CRUD операции с различни видове филтриране и агрегиращи заявки за анализ
    \item Внедряване на механизъм за управление на достъпа чрез роли
    \item Създаване на REST API и уеб интерфейс за демонстрация на функционалностите
\end{enumerate}

Изборът на нерелационна база данни е оправдан от естеството на IoT данните в логистичната сфера. MongoDB предлага следните предимства пред традиционните релационни системи \cite{sadalage2012nosql}:

\begin{itemize}
    \item Гъвкавост в структурата на документите, позволяваща лесно добавяне на нови сензорни типове
    \item Възможност за съхранение на времеви серии от сензорни данни директно в документите
    \item По-добра производителност при работа с големи обеми IoT данни и реално време обработка
    \item Подходящ модел за данни, който естествено отразява обектно-ориентирания подход в IoT системите
\end{itemize}

Проектът демонстрира как нерелационните бази данни могат да бъдат ефективно използвани за решаване на проблеми в IoT и логистични системи, като същевременно предоставя солидна основа за бъдещо разширение и развитие на платформата.