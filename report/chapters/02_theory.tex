% ============================================================================
% Глава 2: Теоретична и технологична обосновка
% ============================================================================

\section{ТЕОРЕТИЧНА И ТЕХНОЛОГИЧНА ОБОСНОВКА}

\subsection{Нерелационни бази данни}

Нерелационните бази данни (NoSQL - Not Only SQL) представляват клас системи за управление на данни, които се различават от традиционните релационни бази данни (RDBMS) по своята архитектура и подход към съхранението и обработката на информацията \cite{sadalage2012nosql}. Основните характеристики на NoSQL системите включват:

\begin{itemize}
    \item \textbf{Липса на фиксирана схема}: За разлика от релационните бази данни, които изискват предварително дефинирана схема с таблици, колони и връзки, NoSQL системите позволяват динамично структуриране на данните. Това дава възможност за лесно добавяне на нови полета и промяна на структурата на документите без необходимост от ALTER TABLE операции.

    \item \textbf{Разпределено съхранение}: NoSQL базите данни са проектирани да работят в разпределени среди, като поддържат хоризонтално мащабиране (scale-out) чрез добавяне на нови сървъри към клъстера. Това позволява обработка на много големи обеми данни и висока наличност.

    \item \textbf{Различни модели на данни}: NoSQL системите предлагат разнообразие от модели за съхранение на данни - документо-ориентиран, колонно-ориентиран, графов, ключ-стойност. Всеки модел е оптимизиран за специфични случаи на използване.
\end{itemize}

\subsection{Документо-ориентиран модел на данни}

За настоящия проект е избран документо-ориентираният модел на NoSQL бази данни, който е реализиран чрез MongoDB. Този модел е най-подходящ за системата за управление на автопарк с IoT интеграция поради следните причини:

\begin{enumerate}
    \item \textbf{Естествено представяне на обектно-ориентирани данни}: IoT сензорите генерират данни с различна структура и вложеност (GPS координати, сензорни метрики, времеви серии). JSON-подобният формат на документите в MongoDB естествено отразява тази сложност без необходимост от нормализация.

    \item \textbf{Гъвкавост при еволюция на схемата}: В IoT системите често се добавят нови типове сензори или се променят характеристиките на съществуващите. Документо-ориентираният модел позволява лесно разширение на структурата на данните без прекъсване на работата на системата.

    \item \textbf{Висока производителност при четене}: Агрегиращите заявки и аналитичните операции, които са критични за системата за управление на флотилии, се изпълняват значително по-бързо в MongoDB благодарение на възможността за съхранение на свързани данни в един документ.

    \item \textbf{Подходящ за времеви серии}: Телеметричните данни представляват времеви серии с висока честота на записване. MongoDB предлага оптимизации за работа с такива данни, включително TTL индекси за автоматично изчистване на стари записи.
\end{enumerate}

\subsection{MongoDB - избрана технология}

MongoDB е водеща документо-ориентирана NoSQL база данни, която предоставя \cite{mongodb_manual}:

\begin{itemize}
    \item \textbf{Документо-ориентирано съхранение}: Данните се съхраняват като BSON (Binary JSON) документи, които могат да съдържат вложени обекти, масиви и различни типове данни.

    \item \textbf{Богат заявков език}: Поддържа сложни заявки с филтриране, сортиране, агрегиране и текстови търсения. Aggregation Framework позволява изграждане на ETL тръбопроводи за трансформация на данни.

    \item \textbf{Индексиране и производителност}: Поддържа различни типове индекси (B-tree, геопространствени, текстови, TTL) за оптимизация на заявките.

    \item \textbf{Репликация и шардинг}: Осигурява висока наличност чрез репликация и хоризонтално мащабиране чрез шардинг.

    \item \textbf{Вградени инструменти за управление}: MongoDB Compass за визуално управление, mongosh за интерактивна работа, и MongoDB Atlas за облачно разполагане.
\end{itemize}

MongoDB е избран поради своята зрялост, активна общност и богат набор от функции, които напълно покриват изискванията на проекта.

\subsection{Използвани технологии}

Проектът е разработен с помощта на съвременни технологии, които осигуряват надеждност, производителност и лесна поддръжка:

\subsubsection{Node.js и Express.js}

За реализацията на приложния слой е избран Node.js - платформа за изпълнение на JavaScript извън браузъра \cite{nodejs_docs}. В комбинация с Express.js framework се осигурява:

\begin{itemize}
    \item \textbf{Неблокиращ I/O}: Асинхронният модел позволява ефективна обработка на множество едновременни връзки, което е критично за IoT системи с висока честота на данни.

    \item \textbf{Единен език за цялата система}: JavaScript се използва както за бекенд, така и за фронтенд, което опростява разработката и поддръжката.

    \item \textbf{Богата екосистема}: NPM предоставя достъп до хиляди модули, включително официалния MongoDB драйвер за Node.js.
\end{itemize}

\subsubsection{Допълнителни технологии}

\begin{itemize}
    \item \textbf{Jest}: Framework за unit тестване, който осигурява 100\% покритие на кода и автоматизирана проверка на функционалностите.

    \item \textbf{Docker}: Контейнеризация на приложението за лесно разполагане и консистентност между различни среди.

    \item \textbf{GitHub Actions}: CI/CD платформа за автоматизирано тестване и деплоймънт при всяка промяна в кода.
\end{itemize}

\subsection{Обосновка на технологичния стек}

Изборът на технологичен стек е направен въз основа на специфичните изисквания на IoT системите за управление на флотилии:

\begin{enumerate}
    \item \textbf{Съответствие с домейна}: MongoDB е широко използвана в IoT приложения поради своята гъвкавост и производителност при работа с времеви серии и сензорни данни.

    \item \textbf{Производителност}: Node.js осигурява висока производителност при обработка на множество едновременни IoT връзки благодарение на event-driven архитектурата.

    \item \textbf{Лесна интеграция}: Всички компоненти (MongoDB, Node.js, Docker) работят добре заедно и имат добра поддръжка в облачни среди.

    \item \textbf{Бъдеща разширяемост}: Архитектурата позволява лесно добавяне на нови сензорни типове, микросървиси и интеграция с облачни платформи.
\end{enumerate}

Тази технологична основа предоставя солидна платформа за разработка на надеждни и мащабируеми IoT системи за логистична индустрия.